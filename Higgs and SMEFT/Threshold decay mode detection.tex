\documentclass[11pt]{article}
\usepackage{enumerate}
\usepackage{tikz,tikz-cd,tikz-3dplot,pgfplots}
\usepackage{amsmath,amsthm,amssymb,amsfonts,amsthm}
\usepackage{mathrsfs}
\usepackage{bm,bbm}
\usepackage{braket}
\usepackage{slashed}
\usepackage{tensor}
\usepackage{indentfirst}
\usepackage[a4paper, total={6.5in, 9in}]{geometry}
\usetikzlibrary{decorations.markings,positioning,decorations.pathmorphing}
\allowdisplaybreaks
\pgfplotsset{width=10cm, compat=1.16}
\renewcommand\bra[1]{{\langle{#1}|}}
\renewcommand\ket[1]{{|{#1}\rangle}}
\renewcommand\bfdefault{b}

\newtheorem{theorem}{Theorem}[section]
\newtheorem{lemma}[theorem]{Lemma}
\newtheorem{corollary}{Corollary}[theorem]
\theoremstyle{definition}
\newtheorem{definition}{Definition}[section]
\theoremstyle{remark}
\newtheorem{remark}{Remark}[section]

\DeclareMathOperator{\sech}{sech}
\DeclareMathOperator{\csch}{csch}
\DeclareMathOperator{\arcsec}{arcsec}
\DeclareMathOperator{\arccot}{arccot}
\DeclareMathOperator{\arccsc}{arccsc}
\DeclareMathOperator{\arccosh}{arccosh}
\DeclareMathOperator{\arcsinh}{arcsinh}
\DeclareMathOperator{\arctanh}{arctanh}
\DeclareMathOperator{\arcsech}{arcsech}
\DeclareMathOperator{\arccsch}{arccsch}
\DeclareMathOperator{\arccoth}{arccoth} 

\begin{document}
	\title{Threshold decay mode detection}
	\author{Seokhyeon Song}
	\maketitle
	
	\section{}
	Dyson resummation gives
	\[\Delta(p^{2})=\frac{i}{p^{2}-M_{0}^{2}+\Pi(p^{2})},\]
	where the self-energy $\Pi(p^{2})$ contains the sum of 1PI loop integrals.
	Following the ``real on-shell'' renormalisation convention, the propagator is
	\[\Delta(p^{2})=\frac{i}{p^{2}-M^{2}+\Pi(p^{2})}=\frac{iZ}{p^{2}-M^{2}+iZ\,\mathrm{Im}\,\Pi(p^{2})+O((p^{2}-M^{2})^{2})},\]
	where the real on-shell mass $M$ and the field-strength renormalisation $Z$ satisfy
	\begin{align*}
		M^{2}&=M_{0}^{2}-\mathrm{Re}\,\Pi(M^{2}),\\
		Z^{-1}&=1+\mathrm{Re}\bigg[\frac{d\Pi(p^{2})}{d(p^{2})}\bigg]_{p^{2}=M^{2}}.
	\end{align*}
	(Particles near threshold ref. here)
	The imaginary part of the denominator of the propagator is related to the particle's total decay width $\Gamma$, where
	\[M\Gamma=Z\,\mathrm{Im}\,\Pi(M^{2}).\]
	Because of unitarity, this definition of decay width coincides with the definition given by the scattering amplitude with one initial state.
	
	One of the alternative definition...
	Pole mass = $M_{p}-\frac{i}{2}\Gamma$...
	But more than one pole or zero pole (depend on the choice convention), so we don't use this definition.
	Maybe some pole trajectory plot here?
\end{document}