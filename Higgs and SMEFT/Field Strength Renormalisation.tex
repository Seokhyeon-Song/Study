\documentclass[11pt]{article}
\usepackage{enumerate}
\usepackage{tikz,tikz-cd,tikz-3dplot,pgfplots}
\usepackage{amsmath,amsthm,amssymb,amsfonts,amsthm}
\usepackage{mathrsfs}
\usepackage{bm,bbm}
\usepackage{braket}
\usepackage{slashed}
\usepackage{tensor}
\usepackage{indentfirst}
\usepackage[a4paper, total={6.5in, 9in}]{geometry}
\usetikzlibrary{decorations.markings,positioning,decorations.pathmorphing}
\allowdisplaybreaks
\pgfplotsset{width=10cm, compat=1.16}
\renewcommand\bra[1]{{\langle{#1}|}}
\renewcommand\ket[1]{{|{#1}\rangle}}
\renewcommand\bfdefault{b}

\newtheorem{theorem}{Theorem}[section]
\newtheorem{lemma}[theorem]{Lemma}
\newtheorem{corollary}{Corollary}[theorem]
\theoremstyle{definition}
\newtheorem{definition}{Definition}[section]
\theoremstyle{remark}
\newtheorem{remark}{Remark}[section]

\DeclareMathOperator{\sech}{sech}
\DeclareMathOperator{\csch}{csch}
\DeclareMathOperator{\arcsec}{arcsec}
\DeclareMathOperator{\arccot}{arccot}
\DeclareMathOperator{\arccsc}{arccsc}
\DeclareMathOperator{\arccosh}{arccosh}
\DeclareMathOperator{\arcsinh}{arcsinh}
\DeclareMathOperator{\arctanh}{arctanh}
\DeclareMathOperator{\arcsech}{arcsech}
\DeclareMathOperator{\arccsch}{arccsch}
\DeclareMathOperator{\arccoth}{arccoth} 

\begin{document}
	Main reference: Peskin \& Schroeder QFT and Srednicki QFT
	\section{Formalism}
	Spectral representation
	\[\int d^{4}x\,e^{ip\cdot x}\bra{\Omega}T\phi(x)\phi(y)\ket{\Omega}=\frac{iZ_{r}}{p^{2}-m^{2}+i\epsilon}+\int_{4m^{2}}^{\infty}\frac{dM^{2}}{2\pi}\rho(M^{2})\frac{\rho(M^{2})}{p^{2}-M^{2}+i\epsilon}.\]
	We may rescale the field so that $\phi=Z_{\phi}^{1/2}\phi_{r}$.
	If we had started with the bare Lagrangian
	\[\mathcal{L}=\frac{1}{2}(\partial_{\mu}\phi)^{2}-\frac{1}{2}m_{0}^{2}\phi^{2}-g_{0}\phi\mathcal{O}+\dots,\]
	where $\mathcal{O}$ is an operator which does not depend on $\phi$, then
	\begin{align*}
		\mathcal{L}&=\frac{1}{2}Z_{\phi}(\partial_{\mu}\phi_{r})^{2}-\frac{1}{2}Z_{\phi}m_{0}^{2}\phi_{r}^{2}-g_{0}Z_{\phi}^{1/2}\phi_{r}\mathcal{O}+\dots\\
		&=\frac{1}{2}(\partial_{\mu}\phi_{r})^{2}-\frac{1}{2}m^{2}\phi_{r}^{2}-g\phi_{r}\mathcal{O}+\frac{1}{2}\delta_{Z}(\partial_{\mu}\phi_{r})^{2}-\frac{1}{2}\delta_{m}\phi_{r}^{2}-\delta_{g}\phi_{r}\mathcal{O}+\dots,
	\end{align*}
	for
	\[\delta_{Z}=Z_{\phi}-1,\quad\delta_{m}=Z_{\phi}m_{0}^{2}-m^{2},\quad\delta_{g}=Z_{\phi}^{1/2}g_{0}-g.\]
	We fix the counterterms by requiring that the exact propagator has a pole $iZ_{r}/(p^{2}-m^{2})$:
	\[\Sigma(m^{2})=0,\quad\Sigma'(m^{2})=Z_{r}^{-1}-1.\]
	From LSZ, we have
	\[\bra{p}S\ket{p}=Z_{r}\times(\text{Diagram}).\]
	
	\section{Specific Example: Scalar Loop}
	The scalar loop self-energy is
	\begin{align*}
		\Sigma(p^{2})&=\frac{\tilde{g}^{2}}{16\pi^{2}}\left(\epsilon^{-1}-\gamma+\int_{0}^{1}dx\,\ln\frac{4\pi}{\Delta}\right)-(Ap^{2}+BM^{2})+O(g^{4})\\
		&=\frac{g^{2}}{16\pi^{2}}\left(\epsilon^{-1}+\int_{0}^{1}dx\,\ln\frac{\mu^{2}}{\Delta}\right)-(Ap^{2}+BM^{2})+O(g^{4}),
	\end{align*}
	where $\Delta=-x(1-x)p^{2}+m^{2}$.
	In the $\overline{\mathrm{MS}}$ scheme, we just cancel the $\epsilon^{-1}$ divergence, so that
	\[\Sigma(p^{2})=\frac{g^{2}}{16\pi^{2}}\int_{0}^{1}dx\,\ln\frac{\mu^{2}}{\Delta}+O(g^{4}).\]
	For now, we take $2m>M$, so that the external particle is stable.
	The physical mass $M_{p}$ is defined by $\Sigma(M_{p}^{2})=M^{2}-M_{p}^{2}$, thus
	\begin{align*}
		M_{p}^{2}&=M^{2}-\Sigma(M_{p}^{2})\\
		&=M^{2}-\Sigma(M^{2})+O(g^{4})\\
		&=M^{2}-\frac{g^{2}}{16\pi^{2}}\int_{0}^{1}dx\,\ln\frac{\mu^{2}}{\Delta_{M}}+O(g^{4}).
	\end{align*}
	For the residue $Z_{r}$, we have
	\begin{align*}
		Z_{r}^{-1}&=1+\Sigma'(M_{p}^{2})\\
		&=1+\Sigma'(M^{2})+O(g^{4})\\
		&=1+\frac{g^{2}}{16\pi^{2}}\int_{0}^{1}dx\,\frac{x(1-x)}{\Delta_{M}}+O(g^{4}).
	\end{align*}
	The physical quantity (somewhat associated with the 1-to-1 scattering amplitude?) is
	\begin{align*}
		Z_{r}(p^{2}-M^{2}+\Sigma(p^{2}))&=\left[1-\frac{g^{2}}{16\pi^{2}}\int_{0}^{1}dx\,\frac{x(1-x)}{\Delta_{M}}+O(g^{4})\right]\left[p^{2}-m^{2}+\frac{g^{2}}{16\pi^{2}}\int_{0}^{1}dx\,\ln\frac{\mu^{2}}{\Delta}+O(g^{4})\right]\\
		&=p^{2}-M^{2}+\frac{g^{2}}{16\pi^{2}}\left[-(p^{2}-M^{2})\int_{0}^{1}dx\,\frac{x(1-x)}{\Delta_{M}}+\int_{0}^{1}dx\,\ln\frac{\mu^{2}}{\Delta}\right]+O(g^{4})\\
		&=p^{2}-M_{p}^{2}+\frac{g^{2}}{16\pi^{2}}\left[-(p^{2}-M^{2})\int_{0}^{1}dx\,\frac{x(1-x)}{\Delta_{M}}+\int_{0}^{1}dx\,\ln\frac{\Delta_{M}}{\Delta}\right]+O(g^{4}).
	\end{align*}
	How about the on-shell scheme?
	In $\mathrm{OS}$, we put $M=M_{p}$ and $Z_{r}=1$.
	So we get
	\[\Sigma(p^{2})=\frac{g^{2}}{16\pi^{2}}\left[-(p^{2}-M_{p}^{2})\int_{0}^{1}dx\,\frac{x(1-x)}{\Delta_{M_{p}}}+\int_{0}^{1}dx\,\ln\frac{\Delta_{M_{p}}}{\Delta}\right]+O(g^{4}).\]
	The physical quantity is
	\[p^{2}-m^{2}+\Sigma(p^{2})=p^{2}-M_{p}^{2}+\frac{g^{2}}{16\pi^{2}}\left[-(p^{2}-M_{p}^{2})\int_{0}^{1}dx\,\frac{x(1-x)}{\Delta_{M_{p}}}+\int_{0}^{1}dx\,\ln\frac{\Delta_{M_{p}}}{\Delta}\right]+O(g^{4}).\]
	We can see that the result is almost the same for the $\overline{\mathrm{MS}}$ and the $\mathrm{OS}$ scheme.
	Only difference is the mass parameter that is used inside the bracket, but its effect is $O(g^{2})$ so the overall quantity is the same up to $O(g^{4})$.
	
	\section{Unstable Particle}
	For unstable particles, the pole mass $M_{p}$ defined by $M_{p}^{2}+\Sigma(M_{p}^{2})=M^{2}$ is complex.
	Hence, the real physical mass would be $M_{R}=\mathrm{Re}\,M_{p}$.
	
	One of the convention in on-shell scheme of unstable particle, we choose $M=M_{\mathrm{OS}}$ so that $\mathrm{Re}\,\Sigma_{\mathrm{OS}}(M_{\mathrm{OS}}^{2})=0$.
	This gives
	\[\Sigma_{\mathrm{OS}}(p^{2})=\frac{g^{2}}{16\pi^{2}}\left[-(p^{2}-M_{\mathrm{OS}}^{2})\mathrm{Re}\,\int_{0}^{1}dx\,\frac{x(1-x)}{\Delta_{M_{\mathrm{OS}}}}+\mathrm{Re}\,\int_{0}^{1}dx\,\ln\frac{\Delta_{M_{\mathrm{OS}}}}{\Delta}+\pi i\int_{0}^{1}dx\,\Theta(-\Delta)\right]+O(g^{4}).\]
	Therefore, the pole mass can be determined with
	\begin{align*}
		M_{p}^{2}&=M_{\mathrm{OS}}^{2}-\Sigma_{\mathrm{OS}}(M_{p}^{2})=M_{\mathrm{OS}}^{2}-\Sigma_{\mathrm{OS}}(M_{\mathrm{OS}}^{2})+O(g^{4})\\
		&=M_{\mathrm{OS}}^{2}-\frac{ig^{2}}{16\pi}\int_{0}^{1}dx\,\Theta(-\Delta_{M_{\mathrm{OS}}})+O(g^{4}).
	\end{align*}
	In this sense, $M_{\mathrm{OS}}=M_{R}+O(g^{4})$.
	We also had the relation
	\[M_{p}^{2}=M_{\overline{\mathrm{MS}}}^{2}-\frac{g^{2}}{16\pi^{2}}\int_{0}^{1}dx\,\ln\frac{\mu^{2}}{\Delta_{M_{\overline{\mathrm{MS}}}}}+O(g^{4}),\]
	so
	\[M_{\overline{\mathrm{MS}}}^{2}=M_{R}^{2}+\frac{g^{2}}{16\pi^{2}}\mathrm{Re}\int_{0}^{1}dx\,\ln\frac{\mu^{2}}{\Delta_{M_{R}}}+O(g^{4}).\]
	Finally, the on-shell pole residue on $p^{2}=M_{p}^{2}$ is
	\begin{align*}
		Z_{r}^{-1}&=1+\Sigma_{\mathrm{OS}}'(M_{p}^{2})\\
		&=1+\frac{g^{2}}{16\pi^{2}}\left[-\mathrm{Re}\,\int_{0}^{1}dx\,\frac{x(1-x)}{\Delta_{M_{\mathrm{OS}}}}+\int_{0}^{1}dx\,\frac{x(1-x)}{\Delta_{M_{p}}}\right]+O(g^{4})\\
		&=1+\frac{ig^{2}}{16\pi^{2}}\mathrm{Im}\,\int_{0}^{1}dx\,\frac{x(1-x)}{\Delta_{M_{R}}}+O(g^{4}).
	\end{align*}
	
	\subsection{Explicit Decay Width}
	Consider the self-energy with an explicit constant decay term:
	\[\Sigma(p^{2})=iM\Gamma+\frac{g^{2}}{16\pi^{2}}\left(\epsilon^{-1}+\int_{0}^{1}dx\,\ln\frac{\mu^{2}}{\Delta}\right)-(Ap^{2}+BM^{2})+O(g^{4}).\]	
	
	We can think of the following three renormalisation schemes.
	\begin{enumerate}
		\item $\overline{\mathrm{MS}}$:
		\[\Sigma_{\overline{\mathrm{MS}}}(p^{2})=iM_{\overline{\mathrm{MS}}}\Gamma_{\overline{\mathrm{MS}}}+\frac{g^{2}}{16\pi^{2}}\int_{0}^{1}dx\,\ln\frac{\mu^{2}}{\Delta}+O(g^{4}).\]
		
		\item Real on-shell:
		\[\Sigma_{\mathrm{OS}}(p^{2})=iM_{\mathrm{OS}}\Gamma_{\mathrm{OS}}+\frac{g^{2}}{16\pi^{2}}\bigg[\begin{aligned}[t]
			&-(p^{2}-M_{\mathrm{OS}}^{2})\mathrm{Re}\,\int_{0}^{1}dx\,\frac{x(1-x)}{\Delta_{\sqrt{M_{\mathrm{OS}}^{2}-iM_{\mathrm{OS}}\Gamma_{\mathrm{OS}}}}}\\
			&+\mathrm{Re}\int_{0}^{1}dx\,\ln\frac{\Delta_{\sqrt{M_{\mathrm{OS}}^{2}-iM_{\mathrm{OS}}\Gamma_{\mathrm{OS}}}}}{\Delta}+i\mathrm{Im}\int_{0}^{1}dx\,\ln\frac{1}{\Delta}\bigg]+O(g^{4}).\\
		\end{aligned}\]
		
		\item Complex on-shell:
		\[\Sigma_{p}(p^{2})=\frac{g^{2}}{16\pi^{2}}\left[-(p^{2}-M_{p}^{2})\int_{0}^{1}dx\,\frac{x(1-x)}{\Delta_{M_{p}}}+\int_{0}^{1}dx\,\ln\frac{\Delta_{M_{p}}}{\Delta}\right]+O(g^{4}).\]
	\end{enumerate}
	We must match the (complex) pole mass:
	\begin{align*}
		M_{p}^{2}&=M_{\overline{\mathrm{MS}}}^{2}-iM_{\overline{\mathrm{MS}}}\Gamma_{\overline{\mathrm{MS}}}-\frac{g^{2}}{16\pi^{2}}\int_{0}^{1}dx\,\ln\frac{\mu^{2}}{\Delta_{\sqrt{M_{\overline{\mathrm{MS}}}^{2}-iM_{\overline{\mathrm{MS}}}\Gamma_{\overline{\mathrm{MS}}}}}}+O(g^{4})\\
		&=M_{\mathrm{OS}}^{2}-iM_{\mathrm{OS}}\Gamma_{\mathrm{OS}}-\frac{ig^{2}}{16\pi^{2}}\bigg[\begin{aligned}[t]
			M_{\mathrm{OS}}\Gamma_{\mathrm{OS}}\mathrm{Re}\,\int_{0}^{1}dx\,\frac{x(1-x)}{\Delta_{\sqrt{M_{\mathrm{OS}}^{2}-iM_{\mathrm{OS}}\Gamma_{\mathrm{OS}}}}}&\\
			+\mathrm{Im}\int_{0}^{1}dx\,\ln\frac{1}{\Delta_{\sqrt{M_{\mathrm{OS}}^{2}-iM_{\mathrm{OS}}\Gamma_{\mathrm{OS}}}}}&\bigg]+O(g^{4}).\\
		\end{aligned}
	\end{align*}
	Hence,
	\[M_{\overline{\mathrm{MS}}}^{2}=M_{\mathrm{OS}}^{2}+\frac{g^{2}}{16\pi^{2}}\mathrm{Re}\int_{0}^{1}dx\,\ln\frac{\mu^{2}}{\Delta_{\sqrt{M_{\mathrm{OS}}^{2}-iM_{\mathrm{OS}}\Gamma_{\mathrm{OS}}}}}+O(g^{4})\]
	and
	\[\Gamma_{\overline{\mathrm{MS}}}=\frac{M_{\mathrm{OS}}}{M_{\overline{\mathrm{MS}}}}\Gamma_{\mathrm{OS}}\left(1+\frac{g^{2}}{16\pi^{2}}\mathrm{Re}\int_{0}^{1}dx\frac{x(1-x)}{\Delta_{\sqrt{M_{\mathrm{OS}}^{2}-iM_{\mathrm{OS}}\Gamma_{\mathrm{OS}}}}}\right)+O(g^{4}).\]
	
	Then we have to evaluate pole residue:
	\begin{align*}
		Z_{\overline{\mathrm{MS}}}&=1-\frac{g^{2}}{16\pi^{2}}\int_{0}^{1}dx\frac{x(1-x)}{\Delta_{\sqrt{M_{\overline{\mathrm{MS}}}^{2}-iM_{\overline{\mathrm{MS}}}\Gamma_{\overline{\mathrm{MS}}}}}}+O(g^{4}),\\
		Z_{\mathrm{OS}}&=1-\frac{ig^{2}}{16\pi^{2}}\mathrm{Im}\int_{0}^{1}dx\frac{x(1-x)}{\Delta_{\sqrt{M_{\mathrm{OS}}^{2}-iM_{\mathrm{OS}}\Gamma_{\mathrm{OS}}}}}+O(g^{4}),\\
		Z_{p}&=1+O(g^{4}).
	\end{align*}
	Note that the $g^{2}$ term in the real on-shell residue is purely imaginary again.
	
	\section{General Formula}
	Let
	\[\Sigma_{\overline{\mathrm{MS}}}(p^{2})=iM_{\overline{\mathrm{MS}}}\Gamma_{\overline{\mathrm{MS}}}+\alpha f(p^{2})+O(\alpha^{2})\]
	for the loop factor $\alpha$.
	The real on-shell self-energy is
	\[\Sigma_{\mathrm{OS}}(p^{2})=iM_{\mathrm{OS}}\Gamma_{\mathrm{OS}}+\alpha\left[f(p^{2})-\mathrm{Re}\,f(M_{\mathrm{OS}}^{2}-iM_{\mathrm{OS}}\Gamma_{\mathrm{OS}})-(p^{2}-M_{\mathrm{OS}}^{2})\mathrm{Re}\,f'(M_{\mathrm{OS}}^{2}-iM_{\mathrm{OS}}\Gamma_{\mathrm{OS}})\right]+O(\alpha^{2}),\]
	and the complex on-shell self-energy is
	\[\Sigma_{p}(p^{2})=\alpha\left[f(p^{2})-f(M_{p}^{2})-(p^{2}-M_{p}^{2})f'(M_{p}^{2})\right]+O(\alpha^{2}).\]
	The pole mass can be found by
	\begin{align*}
		M_{p}^{2}&=M_{\overline{\mathrm{MS}}}^{2}-iM_{\overline{\mathrm{MS}}}\Gamma_{\overline{\mathrm{MS}}}-\alpha f(M_{\overline{\mathrm{MS}}}^{2}-iM_{\overline{\mathrm{MS}}}\Gamma_{\overline{\mathrm{MS}}})+O(\alpha^{2})\\
		&=M_{\mathrm{OS}}^{2}-iM_{\mathrm{OS}}\Gamma_{\mathrm{OS}}-i\alpha\left[\mathrm{Im}\,f(M_{\mathrm{OS}}^{2}-iM_{\mathrm{OS}}\Gamma_{\mathrm{OS}})+M_{\mathrm{OS}}\Gamma_{\mathrm{OS}}\mathrm{Re}\,f'(M_{\mathrm{OS}}^{2}-iM_{\mathrm{OS}}\Gamma_{\mathrm{OS}})\right]+O(\alpha^{2}),
	\end{align*}
	so
	\[M_{\overline{\mathrm{MS}}}^{2}=M_{\mathrm{OS}}^{2}+\alpha\mathrm{Re}\,f(M_{\mathrm{OS}}^{2}-iM_{\mathrm{OS}}\Gamma_{\mathrm{OS}})+O(\alpha^{2})\]
	and
	\[\Gamma_{\overline{\mathrm{MS}}}=\frac{M_{\mathrm{OS}}}{M_{\overline{\mathrm{MS}}}}\Gamma_{\mathrm{OS}}\left[1+\alpha\mathrm{Re}\,f'(M_{\mathrm{OS}}^{2}-iM_{\mathrm{OS}}\Gamma_{\mathrm{OS}})\right]+O(\alpha^{2}).\]
	Finally, the pole residue are
	\begin{align*}
		Z_{\overline{\mathrm{MS}}}&=1-\alpha f'(M_{\overline{\mathrm{MS}}}^{2}-iM_{\overline{\mathrm{MS}}}\Gamma_{\overline{\mathrm{MS}}})+O(\alpha^{2}),\\
		Z_{\mathrm{OS}}&=1-i\alpha\mathrm{Im}\,f'(M_{\mathrm{OS}}^{2}-iM_{\mathrm{OS}}\Gamma_{\mathrm{OS}})+O(\alpha^{2}).
	\end{align*}
\end{document}